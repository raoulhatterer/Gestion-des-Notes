% Created 2020-02-23 Sun 19:38
% Intended LaTeX compiler: pdflatex
\documentclass[11pt]{article}
\usepackage[utf8]{inputenc}
\usepackage[T1]{fontenc}
\usepackage{graphicx}
\usepackage{grffile}
\usepackage{longtable}
\usepackage{wrapfig}
\usepackage{rotating}
\usepackage[normalem]{ulem}
\usepackage{amsmath}
\usepackage{textcomp}
\usepackage{amssymb}
\usepackage{capt-of}
\usepackage{hyperref}
\usepackage{minted}
\usepackage[frenchb]{babel}
\usepackage[x11names]{xcolor}
\hypersetup{linktoc = all, colorlinks = true, urlcolor = DodgerBlue4, citecolor = PaleGreen1, linkcolor = black}
\author{Raoul HATTERER}
\date{\today}
\title{Gestion des notes\\\medskip
\large (Projet DIU)}
\hypersetup{
 pdfauthor={Raoul HATTERER},
 pdftitle={Gestion des notes},
 pdfkeywords={},
 pdfsubject={},
 pdfcreator={Emacs 26.3 (Org mode 9.1.9)}, 
 pdflang={Frenchb}}
\begin{document}

\maketitle
\setcounter{tocdepth}{2}
\tableofcontents

\newpage


\section{Installations et paramétrages}
\label{sec:org6cf1539}
\textbf{Remarque}: Certaines parties concernent plus spécifiquement mon utilisation personnelle avec \href{https://emacsformacosx.com}{emacs sous OSX}.
\subsection{Prérequis}
\label{sec:org2367d20}
\subsubsection{Installer Homebrew}
\label{sec:org125cf76}

Pour installer MariaDB, il faut disposer de \href{https://brew.sh/index\_fr}{Homebrew} (The Missing Package Manager for macOS).

\subsubsection{Installer MariaDB}
\label{sec:orgdcf676c}

MariaDB Server peut ensuite être installé grâce à cette commande:

\begin{minted}[,frame=single, label = \textrm{\textbf{shell}}, labelposition = topline, samepage = t]{shell}
brew update
brew install mariadb
\end{minted}

\subsubsection{Démarrer MariaDB Server}
\label{sec:org0c67830}
Après installation, on peut (mais on ne va pas le faire) démarrer le serveur avec la commande:
\begin{minted}[,frame=single, label = \textrm{\textbf{shell}}, labelposition = topline, samepage = t]{shell}
mysql.server start
\end{minted}

Il est plus commode d'activer le démarrage automatique du serveur:

\begin{minted}[,frame=single, label = \textrm{\textbf{shell}}, labelposition = topline, samepage = t]{shell}
brew services start mariadb
\end{minted}

\subsubsection{Vérifier que MariaDB a démarré}
\label{sec:org69ed688}

\begin{minted}[,frame=single, label = \textrm{\textbf{shell}}, labelposition = topline, samepage = t]{shell}
brew services list
\end{minted}

\subsubsection{Client-serveur}
\label{sec:org3de0c88}

Comme la plupart des SGBD ([S]ystème de [G]estion de [B]ase de [D]onnées), MariaDB est basée sur le \textbf{modèle client-serveur}. Cela implique que la base de données se trouve sur un \textbf{serveur} auquel nous n'avons pas accès directement. Il faut passer par un \textbf{client} qui fera la liaison entre nous et le serveur.

Donc, maintenant que  MariaDB Server a démarré, on va s'y connecter en utilisant un logiciel client (nommé \texttt{mysql})\ldots{} mais sous quelle identité et avec quels droits?

\subsubsection{Distinguer les utilisateurs}
\label{sec:org2641309}

\begin{enumerate}
\item Types d'utilisateurs
\label{sec:orgc456a68}

Il faut distinguer trois types d'utilisateur d'un SGBD ayant chacun des rôles bien distincts :
\begin{enumerate}
\item \textbf{L'administrateur} du SGBD et donc des Bases de Données. Il s'appelle \texttt{root}. Son rôle est d'installer et gérer le SGBD dans sa globalité. Chaque fois que l'administrateur crée une base de données, il délègue sa gestion à un autre utilisateur à qui il donne les droits nécessaires. \textbf{Remarque:}  Il ne faut pas travailler sur une bases de données en tant que \texttt{root}; ce n'est pas son rôle (\href{https://www.youtube.com/watch?v=oiQG6tP3940}{c'est mal}).
\item \textbf{Le gestionnaire} de Bases de Données (dans un SGBD existant) qui doit connaître  \textbf{sql} (Structured Query Language) : création (CREATE TABLE), modification (ALTER TABLE), suppression (DROP TABLE) de tables\ldots{}
\item \textbf{L'utilisateur final} de la base de donnée. Il va interagir avec tout ou partie de la base de donnée à travers une application se voulant conviviale; il n'a pas à connaître \textbf{sql}.
\end{enumerate}



\item À la création
\label{sec:org84e5f19}

\textbf{À la création, la base de donnée est initialisée avec 2 comptes:} 
\begin{itemize}
\item le premier correspond au nom d'utilisateur actuel (checké avec la commande \texttt{whoami}); il ne nécessite pas de mot de passe;
\item tandis que le second est le compte administrateur \texttt{root} ; il est inaccessible tant qu'on n'a pas défini de mot de passe.
\end{itemize}

On va donc faire ce qu'il faut pour activer l'accès au compte administrateur (root) puis s'y connecter pour créer une base de données (\texttt{bd\_gestion\_des\_notes}) et un compte gestionnaire.  

\item Activer l'accès au compte \texttt{root}
\label{sec:org758eaa0}

Pour cela, d'abord se connecter au client \textbf{mysql}  avec son nom d'utilisateur:

\begin{minted}[,frame=single, label = \textrm{\textbf{shell}}, labelposition = topline, samepage = t]{shell}
mysql -u $(whoami)
\end{minted}

Puis, grâce au commandes du client \textbf{mysql}, définir le mot de passe de \texttt{root}:

\begin{minted}[,frame=single, label = \textrm{\textbf{sql}}, labelposition = topline, samepage = t]{sql}
SHOW databases;
USE mysql;
SET password FOR 'root'@'localhost' = 
password('YOUR_ROOT_PASSWORD_HERE');
FLUSH PRIVILEGES;
quit
\end{minted}

où \texttt{YOUR\_ROOT\_PASSWORD\_HERE} est à remplacer par le mot de passe souhaité.
\end{enumerate}

\subsection{Création par \texttt{root} d'une nouvelle base de données et d'un gestionnaire pour assurer sa gestion}
\label{sec:orgf7231b2}

\subsubsection{Login en tant que \texttt{root}}
\label{sec:org77d83de}

Maintenant, nous pouvons nous connecter en tant que \texttt{root}:

\begin{minted}[,frame=single, label = \textrm{\textbf{shell}}, labelposition = topline, samepage = t]{shell}
mysql -u root -p
\end{minted}

Il faudra fournir le mot de passe renseigné à l'étape précédente.

\subsubsection{Commandes de base à la disposition de l'administrateur}
\label{sec:orgeed9823}

\begin{itemize}
\item Afficher le statut du serveur:\\

\begin{minted}[,frame=single, label = \textrm{\textbf{sql}}, labelposition = topline, samepage = t]{sql}
SHOW STATUS;
\end{minted}

\item Créer une nouvelle base de données:\\

\begin{minted}[,frame=single, label = \textrm{\textbf{sql}}, labelposition = topline, samepage = t]{sql}
CREATE DATABASE nom_database;
\end{minted}

\item Lister les bases de données:\\

\begin{minted}[,frame=single, label = \textrm{\textbf{sql}}, labelposition = topline, samepage = t]{sql}
SHOW DATABASES;
\end{minted}

\item Effacer une base de donnée:\\

\begin{minted}[,frame=single, label = \textrm{\textbf{sql}}, labelposition = topline, samepage = t]{sql}
DROP DATABASE nom_database;
\end{minted}
\end{itemize}





\subsubsection{Création de la nouvelle base de données \texttt{bd\_gestion\_des\_notes}:}
\label{sec:org0943a2b}
\begin{minted}[,frame=single, label = \textrm{\textbf{sql}},]{sql}
CREATE DATABASE bd_gestion_des_notes;
SHOW DATABASES;
\end{minted}

\subsubsection{Création d'un utilisateur \texttt{gestionnaire}}
\label{sec:org5da28a7}
Utilisateur à qui \texttt{root} va donner les droits nécessaires pour créer ou modifier des tables dans la base de données \texttt{bd\_gestion\_des\_notes} ou pour inscrire des utilisateurs
\begin{minted}[,frame=single, label = \textrm{\textbf{sql}},  labelposition = topline, samepage = t]{sql}
CREATE USER user_gestionnaire@localhost
IDENTIFIED BY 'gestionnaire';
GRANT ALL PRIVILEGES ON bd_gestion_des_notes.* 
TO user_gestionnaire@localhost;
quit
\end{minted}

\subsubsection{Effacer (si besoin) l'utilisateur \texttt{gestionnaire}}
\label{sec:org591fd17}

\begin{minted}[,frame=single, label = \textrm{\textbf{sql}},  labelposition = topline, samepage = t]{sql}
DROP USER user_gestionnaire@localhost;
\end{minted}


\section{Afficher les utilisateurs}
\label{sec:org74c73bf}
\subsection{Afficher tous les utilisateurs (en tant qu'administrateur)}
\label{sec:org2237326}

Connecté en \texttt{root} saisir la commande:
\begin{minted}[,frame=single, label = \textrm{\textbf{sql}},  labelposition = topline, samepage = t]{sql}
select host, user from mysql.user;
\end{minted}

\subsection{Afficher l'utilisateur connecté}
\label{sec:orgb991ad4}

Deux commandes possibles pour cela:
\begin{minted}[,frame=single, label = \textrm{\textbf{sql}},  labelposition = topline, samepage = t]{sql}
SELECT current_user;
\end{minted}

ou 

\begin{minted}[,frame=single, label = \textrm{\textbf{sql}},  labelposition = topline, samepage = t]{sql}
SELECT user();
\end{minted}


\section{Gestion de la base de données}
\label{sec:orgfbb408c}

\subsection{Se connecter au client \texttt{mysql} en tant que gestionnaire}
\label{sec:orgfa648fc}

\begin{minted}[,frame=single, label = \textrm{\textbf{sql}},  labelposition = topline, samepage = t]{sql}
mysql -u user_gestionnaire -p
\end{minted}

Saisir le mot de passe précédemment défini. 

\subsection{Lister les bases de données (auxquelles le gestionnaire a accès)}
\label{sec:org9ea8a51}

\begin{minted}[,frame=single, label = \textrm{\textbf{sql}},  labelposition = topline, samepage = t]{sql}
SHOW DATABASES;
\end{minted}

\subsection{Se connecter à une base de données}
\label{sec:orga48b96b}
Pour utiliser la base de donnée \texttt{bd\_gestion\_des\_notes}:

\begin{minted}[,frame=single, label = \textrm{\textbf{sql}},  labelposition = topline, samepage = t]{sql}
USE bd_gestion_des_notes;      
\end{minted}

\subsection{Afficher les tables de la base courante}
\label{sec:org88b4c21}

\begin{minted}[,frame=single, label = \textrm{\textbf{sql}},  labelposition = topline, samepage = t]{sql}
SHOW TABLES;
\end{minted}

\subsection{Afficher la structure de la table}
\label{sec:orgaf64954}

\begin{minted}[,frame=single, label = \textrm{\textbf{sql}},  labelposition = topline, samepage = t]{sql}
DESCRIBLE nom_table;
\end{minted}

\subsection{Créer une nouvelle table}
\label{sec:org9a37a76}

\begin{minted}[,frame=single, label = \textrm{\textbf{sql}},  labelposition = topline, samepage = t]{sql}
CREATE TABLE nom_table (
nom_colonne1 int,
nom_colonne2 VARCHAR(100)
);
\end{minted}

\subsection{Ajouter une colonne à une table}
\label{sec:org47eb1be}
\begin{minted}[,frame=single, label = \textrm{\textbf{sql}},  labelposition = topline, samepage = t]{sql}
ALTER TABLE nom_table ADD nom_colonne type_colonne;
\end{minted}

\subsection{Effacer une colonne}
\label{sec:org93d4b91}
\begin{minted}[,frame=single, label = \textrm{\textbf{sql}},  labelposition = topline, samepage = t]{sql}
ALTER TABLE nom_table DROP COLUMN nom_colonne;
\end{minted}

\subsection{Effacer une table}
\label{sec:org536ac8a}
\begin{minted}[,frame=single, label = \textrm{\textbf{sql}},  labelposition = topline, samepage = t]{sql}
DROP nom_table;
\end{minted}

\subsection{Afficher le contenu de la table}
\label{sec:org96ac0e3}
\begin{minted}[,frame=single, label = \textrm{\textbf{sql}},  labelposition = topline, samepage = t]{sql}
SELECT * FROM nom_table;
\end{minted}


\section{SQL avec emacs}
\label{sec:org8198105}

\subsection{Se connecter au client \texttt{mysql} dans emacs}
\label{sec:orge07b911}

Utiliser la commande: \texttt{M-x sql-mysql}

Renseigner:
\begin{itemize}
\item User: \texttt{user\_gestionnaire}
\item Password: \texttt{gestionnaire}
\item Database : (ne rien mettre)
\item Server: \texttt{localhost}
\end{itemize}

\subsection{Si emacs ne trouve pas le programme mysql}
\label{sec:orgc4f52cd}
\begin{itemize}
\item Déterminer l'emplacement de \texttt{mysql} avec la commande:
\end{itemize}
\begin{minted}[,frame=single, label = \textrm{\textbf{shell}},  labelposition = topline, samepage = t]{shell}
which mysql
\end{minted}
qui retourne \texttt{/usr/local/bin/mysql}

\begin{itemize}
\item modifier \texttt{.emacs.d/init.el}
\end{itemize}

\begin{minted}[,frame=single, label = \textrm{\textbf{elisp}},  labelposition = topline, samepage = t]{elisp}
(add-to-list 'exec-path "/usr/local/bin")
\end{minted}

Emacs recherche les programmes dans les répertoires listés dans la variable \texttt{exec-path}. On a ajouté  \texttt{/usr/local/bin} à cette variable.

\subsection{Faciliter la connexion}
\label{sec:org404b239}
À placer dans \texttt{.emacs.d/init.el}
\begin{minted}[,frame=single, label = \textrm{\textbf{elisp}},  labelposition = topline, samepage = t]{elisp}
(use-package sql
  :ensure t
  :config
  (sql-set-product-feature 'mysql :prompt-regexp 
                 "^\\(MariaDB\\|MySQL\\) \\[[_a-zA-Z]*\\]> ")
  (setq sql-user "user_gestionnaire")
  (setq sql-database "bd_gestion_des_notes")
  (setq sql-server "localhost")
  (define-key comint-mode-map [mouse-3] 'comint-insert-input)
  )
\end{minted}

Connexion acilitée, toujours avec la commande \texttt{M-x sql-mysql} et dorénavant il ne reste plus qu'à saisir le mot de passe utilisateur (en l'occurrence: \texttt{gestionnaire}); le reste étant renseigné à l'avance.

\subsection{Mots clés automatiquement en majuscule}
\label{sec:org35fab77}

SQLUP-MODE permet d'écrire les mots clés SQL en majuscule.

Les lignes suivantes sont  à placer dans le fichier \texttt{init.el}:

\begin{minted}[,frame=single, label = \textrm{\textbf{elisp}},  labelposition = topline, samepage = t]{elisp}
(use-package sqlup-mode
  :ensure t
  :config
  (add-hook 'sql-mode-hook 'sqlup-mode)
  (add-hook 'sql-interactive-mode-hook 'sqlup-mode)
  (add-hook 'redis-mode-hook 'sqlup-mode)
  )
\end{minted}

\subsection{Autocompletion}
\label{sec:org6f91d16}

\begin{minted}[,frame=single, label = \textrm{\textbf{elisp}},  labelposition = topline, samepage = t]{elisp}
;;; SQL COMPLETION
;; put the root password in  ~/.emacs.d/lisp/mysql.el
(require 'sql-completion)
(setq sql-interactive-mode-hook
      (lambda ()
        (define-key sql-interactive-mode-map "\t"
          'comint-dynamic-complete)
        (sql-mysql-completion-init)))

\end{minted}


\section{Exécuter un script sql}
\label{sec:org7c441c1}

Il faut déclarer le fichier SQL comme source:

\begin{minted}[,frame=single, label = \textrm{\textbf{sql}},  labelposition = topline, samepage = t]{sql}
source path_to/the/script_file.sql
\end{minted}

Par exemple:

\begin{minted}[,frame=single, label = \textrm{\textbf{sql}},  labelposition = topline, samepage = t]{sql}
source SOURCE ./initdb_gestiondesnotes.SQL
\end{minted}


\section{Python tkinter table widget for displaying tabular data}
\label{sec:org56bfc8c}

\subsection{installation}
\label{sec:org6b042e2}
\begin{minted}[]{shell}
pip install tksheet
\end{minted}

\subsection{source}
\label{sec:org3c381fd}
\url{https://github.com/ragardner/tksheet}

\subsection{paramètres}
\label{sec:orgfd77081}
\url{https://github.com/ragardner/tksheet/blob/master/DOCUMENTATION.md}

\subsection{fonctions}
\label{sec:orge00e4e5}
\url{https://github.com/ragardner/tksheet/blob/master/tksheet/\_tksheet.py}
\end{document}