% Created 2020-02-20 Thu 01:30
% Intended LaTeX compiler: pdflatex
\documentclass[11pt]{article}
\usepackage[utf8]{inputenc}
\usepackage[T1]{fontenc}
\usepackage{graphicx}
\usepackage{grffile}
\usepackage{longtable}
\usepackage{wrapfig}
\usepackage{rotating}
\usepackage[normalem]{ulem}
\usepackage{amsmath}
\usepackage{textcomp}
\usepackage{amssymb}
\usepackage{capt-of}
\usepackage{hyperref}
\usepackage{minted}
\usepackage[frenchb]{babel}
\usepackage[x11names]{xcolor}
\hypersetup{linktoc = all, colorlinks = true, urlcolor = DodgerBlue4, citecolor = PaleGreen1, linkcolor = black}
\author{Raoul HATTERER}
\date{\today}
\title{Gestion des notes\\\medskip
\large (Projet DIU)}
\hypersetup{
 pdfauthor={Raoul HATTERER},
 pdftitle={Gestion des notes},
 pdfkeywords={},
 pdfsubject={},
 pdfcreator={Emacs 26.3 (Org mode 9.1.9)}, 
 pdflang={Frenchb}}
\begin{document}

\maketitle
\setcounter{tocdepth}{2}
\tableofcontents

\newpage


\section{Installations et paramétrages}
\label{sec:orgb8900ab}
\textbf{Remarque}: Certaines parties concernent plus spécifiquement mon utilisation personnelle avec \href{https://emacsformacosx.com}{emacs sous OSX}.

\subsection{Installer Homebrew}
\label{sec:org8a40175}

Pour installer MariaDB, il faut disposer de \href{https://brew.sh/index\_fr}{Homebrew} (The Missing Package Manager for macOS).

\subsection{Installer MariaDB}
\label{sec:orgfbd4eea}

MariaDB Server peut ensuite être installé grâce à cette commande:

\begin{minted}[,frame=single, label = \textrm{\textbf{shell}}, labelposition = topline, samepage = t]{shell}
brew update
brew install mariadb
\end{minted}

\subsection{Démarrer MariaDB Server}
\label{sec:org37d22bf}
Après installation, on peut (mais on ne va pas le faire) démarrer le serveur avec la commande:
\begin{minted}[,frame=single, label = \textrm{\textbf{shell}}, labelposition = topline, samepage = t]{shell}
mysql.server start
\end{minted}

Il est plus commode d'activer le démarrage automatique du serveur:

\begin{minted}[,frame=single, label = \textrm{\textbf{shell}}, labelposition = topline, samepage = t]{shell}
brew services start mariadb
\end{minted}

\subsection{Vérifier que MariaDB a démarré}
\label{sec:org6cf8bdd}

\begin{minted}[,frame=single, label = \textrm{\textbf{shell}}, labelposition = topline, samepage = t]{shell}
brew services list
\end{minted}

\subsection{Client-serveur}
\label{sec:org7fe9b6e}

Comme la plupart des SGBD ([S]ystème de [G]estion de [B]ase de [D]onnées), MariaDB est basée sur le \textbf{modèle client-serveur}. Cela implique que la base de données se trouve sur un \textbf{serveur} auquel nous n'avons pas accès directement. Il faut passer par un \textbf{client} qui fera la liaison entre nous et le serveur.

Donc, maintenant que  MariaDB Server a démarré, on va s'y connecter en utilisant un logiciel client (nommé \texttt{mysql})\ldots{} mais sous quelle identité et avec quels droits?

\subsection{Distinguer les utilisateurs}
\label{sec:orgb7b8813}

\subsubsection{Types d'utilisateurs}
\label{sec:orgbcd1622}

Il faut distinguer trois types d'utilisateur d'un SGBD ayant chacun des rôles bien distincts :
\begin{enumerate}
\item L'\textbf{administrateur} du SGBD et donc des Bases de Données. Il s'appelle \texttt{root}. Son rôle est d'installer et gérer le SGBD dans sa globalité. Chaque fois que l'administrateur crée une base de données, il délègue sa gestion à un autre utilisateur à qui il donne les droits nécessaires. \textbf{Remarque:}  Il ne faut pas travailler sur une bases de données en tant que \texttt{root}; ce n'est pas son rôle (\href{https://www.youtube.com/watch?v=oiQG6tP3940}{c'est mal}).
\item Le \textbf{gestionnaire} de Bases de Données (dans un SGBD existant) qui doit connaître  \textbf{sql} (Structured Query Language) : création (CREATE TABLE), modification (ALTER TABLE), suppression (DROP TABLE) de tables\ldots{}
\item L'\textbf{utilisateur} de bases de données (il va interagir avec la base de donnée à travers une application se voulant conviviale; il n'a pas à connaître \textbf{sql}).
\end{enumerate}



\subsubsection{À la création}
\label{sec:orgf19e4ac}

\textbf{À la création, la base de donnée est initialisée avec 2 comptes:} 
\begin{itemize}
\item le premier correspond au nom d'utilisateur actuel (checké avec la commande \texttt{whoami}); il ne nécessite pas de mot de passe;
\item tandis que le second est le compte administrateur \texttt{root} ; il est inaccessible tant qu'on n'a pas défini de mot de passe.
\end{itemize}

On va donc faire ce qu'il faut pour activer l'accès au compte administrateur (root) puis s'y connecter pour créer une base de données (\texttt{bd\_gestion\_des\_notes}) et un compte gestionnaire.  

\subsubsection{Activer l'accès au compte \texttt{root}}
\label{sec:org687c8f8}

Pour cela, d'abord se connecter au client \textbf{mysql}  avec son nom d'utilisateur:

\begin{minted}[,frame=single, label = \textrm{\textbf{shell}}, labelposition = topline, samepage = t]{shell}
mysql -u $(whoami)
\end{minted}

Puis, grâce au commandes du client \textbf{mysql}, définir le mot de passe de \texttt{root}:

\begin{minted}[,frame=single, label = \textrm{\textbf{sql}}, labelposition = topline, samepage = t]{sql}
show databases;
use mysql;
SET password FOR 'root'@'localhost' = 
password('YOUR_ROOT_PASSWORD_HERE');
flush PRIVILEGES;
quit
\end{minted}

où \texttt{YOUR\_ROOT\_PASSWORD\_HERE} est à remplacer par le mot de passe souhaité.

\subsection{Création d'une nouvelle base de données et d'un gestionnaire pour assurer sa gestion}
\label{sec:org5429560}

\subsubsection{Login en tant que \texttt{root}}
\label{sec:orgabafa6a}

Maintenant, nous pouvons nous connecter en tant que \texttt{root}:

\begin{minted}[,frame=single, label = \textrm{\textbf{shell}}, labelposition = topline, samepage = t]{shell}
mysql -u root -p
\end{minted}

Il faudra fournir le mot de passe renseigné à l'étape précédente.

\subsubsection{Commandes de base à la disposition de l'administrateur}
\label{sec:orge64b829}

\begin{itemize}
\item Afficher le statut du serveur:\\

\begin{minted}[,frame=single, label = \textrm{\textbf{sql}}, labelposition = topline, samepage = t]{sql}
SHOW STATUS;
\end{minted}

\item Créer une nouvelle base de données:\\

\begin{minted}[,frame=single, label = \textrm{\textbf{sql}}, labelposition = topline, samepage = t]{sql}
CREATE DATABASE nom_database;
\end{minted}

\item Lister les bases de données:\\

\begin{minted}[,frame=single, label = \textrm{\textbf{sql}}, labelposition = topline, samepage = t]{sql}
SHOW DATABASES;
\end{minted}

\item Effacer une base de donnée:\\

\begin{minted}[,frame=single, label = \textrm{\textbf{sql}}, labelposition = topline, samepage = t]{sql}
DROP DATABASE nom_database;
\end{minted}
\end{itemize}





\subsubsection{Création de la nouvelle base de données \texttt{bd\_gestion\_des\_notes}:}
\label{sec:orgaceb3a3}
\begin{minted}[,frame=single, label = \textrm{\textbf{sql}},]{sql}
CREATE DATABASE bd_gestion_des_notes;
SHOW DATABASES;
\end{minted}

\subsubsection{Création d'un utilisateur \texttt{gestionnaire}}
\label{sec:org3f6b739}
Utilisateur à qui \texttt{root} va donner les droits nécessaires pour créer ou modifier des tables dans la base de données \texttt{bd\_gestion\_des\_notes} ou pour inscrire des utilisateurs
\begin{minted}[,frame=single, label = \textrm{\textbf{sql}},]{sql}
CREATE USER user_gestionnaire@localhost
IDENTIFIED BY 'gestionnaire';
GRANT ALL PRIVILEGES ON bd_gestion_des_notes.* 
TO user_gestionnaire@localhost;
quit
\end{minted}

\subsubsection{Effacer l'utilisateur \texttt{gestionnaire}}
\label{sec:org6fb41f8}

\begin{minted}[,frame=single, label = \textrm{\textbf{sql}},]{sql}
DROP USER user_gestionnaire@localhost;
\end{minted}


\section{Afficher les utilisateurs}
\label{sec:orgd7e5647}
\subsection{Afficher tous les utilisateurs (en tant qu'administrateur)}
\label{sec:orgf71ae18}

Connecté en \texttt{root} saisir la commande:
\begin{minted}[,frame=single, label = \textrm{\textbf{sql}},]{sql}
select host, user from mysql.user;
\end{minted}

\subsection{Afficher l'utilisateur connecté}
\label{sec:orgcb42eb0}

Deux commandes possibles pour cela:
\begin{minted}[,frame=single, label = \textrm{\textbf{sql}},]{sql}
SELECT current_user;
\end{minted}

ou 

\begin{minted}[,frame=single, label = \textrm{\textbf{sql}},]{sql}
SELECT user();
\end{minted}


\section{Gestion de la base de données}
\label{sec:org87aa770}

\subsection{Se connecter au client \texttt{mysql} en tant que gestionnaire}
\label{sec:org5ba62e6}

\begin{minted}[,frame=single, label = \textrm{\textbf{sql}},]{sql}
mysql -u user_gestionnaire -p
\end{minted}

Saisir le mot de passe précédemment défini. 


\subsection{Lister les bases de données (auxquelles le gestionnaire a accès):}
\label{sec:orge98eb96}

\begin{minted}[,frame=single, label = \textrm{\textbf{sql}},]{sql}
SHOW DATABASES;
\end{minted}

\subsection{Choisir d'utiliser la base de donnée \texttt{bd\_gestion\_des\_notes}}
\label{sec:orgf05827b}

\begin{minted}[,frame=single, label = \textrm{\textbf{sql}},]{sql}
use bd_gestion_des_notes;      
\end{minted}


\subsection{Afficher les tables de la base courante:}
\label{sec:org388e285}

\begin{minted}[,frame=single, label = \textrm{\textbf{sql}},]{sql}
SHOW TABLES;
\end{minted}

\subsection{Afficher la structure de la table:}
\label{sec:org4d98a60}

\begin{minted}[,frame=single, label = \textrm{\textbf{sql}},]{sql}
DESCRIBLE nom_table;
\end{minted}

\subsection{Créer une nouvelle table:}
\label{sec:orgf75a940}

\begin{minted}[,frame=single, label = \textrm{\textbf{sql}},]{sql}
CREATE TABLE nom_table (
nom_colonne1 int,
nom_colonne2 VARCHAR(100)
);
\end{minted}

\subsection{Ajouter une colonne à une table:}
\label{sec:orgdcd58c5}
\begin{minted}[,frame=single, label = \textrm{\textbf{sql}},]{sql}
ALTER TABLE nom_table ADD nom_colonne type_colonne;
\end{minted}

\subsection{Effacer une colonne:}
\label{sec:org5a24125}
\begin{minted}[,frame=single, label = \textrm{\textbf{sql}},]{sql}
ALTER TABLE nom_table DROP COLUMN nom_colonne;
\end{minted}

\subsection{Effacer une table:}
\label{sec:org15125f3}
\begin{minted}[,frame=single, label = \textrm{\textbf{sql}},]{sql}
DROP nom_table;
\end{minted}

\subsection{Afficher le contenu de la table:}
\label{sec:org636b602}
\begin{minted}[,frame=single, label = \textrm{\textbf{sql}},]{sql}
SELECT * FROM nom_table;
\end{minted}






\section{sql-mode dans emacs}
\label{sec:org6340d34}

\subsection{Se connecter au serveur dans emacs}
\label{sec:org5ffab25}

\texttt{M-x sql-mysql}

Renseigner:
\begin{itemize}
\item User: \texttt{user\_gestionnaire}
\item Password: \texttt{gestionnaire}
\item Database : \texttt{bd\_gestion\_des\_notes}
\item Server: \texttt{localhost}
\end{itemize}

\subsection{Si emacs ne trouve pas le programme mysql}
\label{sec:orga02b682}
\begin{itemize}
\item Déterminer l'emplacement de \texttt{mysql}
\end{itemize}
\begin{minted}[,frame=single, label = \textrm{\textbf{shell}},]{shell}
which mysql
\end{minted}
retourne \texttt{/usr/local/bin/mysql}

\begin{itemize}
\item modifier \texttt{.emacs.d/init.el}
\end{itemize}
Emacs uses the directories listed in the exec-path variable to look for programs. You can add to it like so:

\begin{minted}[,frame=single, label = \textrm{\textbf{elisp}},]{elisp}
(add-to-list 'exec-path "/usr/local/bin")
\end{minted}

\subsection{Connexion facilitée}
\label{sec:orgc8fa61e}
À placer dans \texttt{.emacs.d/init.el}
\begin{minted}[,frame=single, label = \textrm{\textbf{elisp}},]{elisp}
(setq sql-user "user_gestionnaire")
(setq sql-password "gestionnaire")
(setq sql-database "bd_gestion_des_notes")
(setq sql-server "localhost")
\end{minted}

\subsection{SQL-UPCASE}
\label{sec:org1097cb0}
\url{https://www.emacswiki.org/emacs/SqlUpcase}
A minor mode which converts lower-case SQL keywords to upper-case as you type or otherwise insert text in the buffer – for instance, killing and yanking an entire SQL query would upcase all keywords in that query.
Commands are also provided for upcasing all keywords in the buffer (‘sql-upcase-buffer’), or a specified region (‘sql-upcase-region’).
\begin{minted}[]{elisp}
(when (require 'sql-upcase nil :noerror)
  (add-hook 'sql-mode-hook 'sql-upcase-mode)
  (add-hook 'sql-interactive-mode-hook 'sql-upcase-mode))
\end{minted}


\section{How to run sql script}
\label{sec:orgb12d894}

If you’re at the MySQL command line mysql> you have to declare the SQL file as source.

\begin{minted}[]{sql}
mysql> source \path\to\the\script_file.sql;
\end{minted}

For example:

\begin{minted}[]{sql}
mysql> source \home\user\Desktop\script_file.sql;
\end{minted}


\section{Python tkinter table widget for displaying tabular data}
\label{sec:org1a6b6a1}

\subsection{installation}
\label{sec:org2a6dd1e}
\begin{minted}[]{shell}
pip install tksheet
\end{minted}

\subsection{source}
\label{sec:orge85f90b}
\url{https://github.com/ragardner/tksheet}

\subsection{paramètres}
\label{sec:org11498f4}
\url{https://github.com/ragardner/tksheet/blob/master/DOCUMENTATION.md}

\subsection{fonctions}
\label{sec:orgf4e994f}
\url{https://github.com/ragardner/tksheet/blob/master/tksheet/\_tksheet.py}
\end{document}